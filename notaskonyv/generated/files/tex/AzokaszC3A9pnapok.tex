\begin{minipage}{\textwidth}
\cimcentered{Azok a szép napok}{AzokaszC3A9pnapok}
\kottastart
\kottapart{AzokaszC3A9pnapok-1}
\kottapart{AzokaszC3A9pnapok-2}
\kottapart{AzokaszC3A9pnapok-3}
\kottapart{AzokaszC3A9pnapok-4}
\kottaend
\end{minipage}
~\vspace{1.0cm}
\newline
