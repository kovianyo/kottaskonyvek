\usepackage{pdfpages}
% ---------------------------------------------------------
% magyar beállítás
%\usepackage[latin2]{inputenc}		% Latin-2-es kódolású ez a forrás: ékezetek %elehtne még: %\usepackage[utf8]{inputenc}
\usepackage[utf8]{inputenc}
%\usepackage{t1enc}			% a belsõ ábrázolás és a kimeneti font-kódolás  a T1 kódolás % csúnya lesz tõle	
\usepackage[magyar]{babel}  % magyar támogatás

\usepackage{indentfirst}  % to indent the first lines of sections

\sloppy % jobbra kilógás helyett új sorba tördelés
\frenchspacing % nincs nagyobb szóköz a mondatok után

% idézõlejek: ``szem elõtt''

% magyar latex leírás:
% http://www.cs.bme.hu/~egmont/latex/

% http://www.szofi.hu/gnu/magyarispell/
% http://csomalin.elte.hu/~flu/konyvek/lkh_1.html
% http://www.math.bme.hu/latex/

\usepackage[a5paper,inner=1.3cm,outer=1.2cm,top=0.7cm,bottom=0.7cm,pdftex, includeheadfoot]{geometry}

\setlength\headheight{5pt} 

\setlength\parindent{0in}

\usepackage{graphicx}
\usepackage{alltt}

\usepackage{fancyhdr}
\usepackage{chngpage}

% elválasztás kikapcsolása
\usepackage[none]{hyphenat} 

\usepackage{makeidx}
\makeindex

\usepackage{extramarks}

%\usepackage{multind}
%\makeindex{szavak}

%\usepackage{times} % Times New Roman betûtípus

\usepackage{multicol}

% space before and after a multicol environment
\setlength{\multicolsep}{0pt}

\usepackage{extramarks}

\usepackage{pdflscape}
%\usepackage[landscape]{geometry} % all pages landscape

% for \textsubscript{} :
\usepackage{fixltx2e}

% for \textmusicalnote
\usepackage{textcomp}

\setlength\fboxsep{0pt}
\setlength\fboxrule{0.5pt}

% usage: \incpicture{filename}{title}{graphpar(scale=0.5)}
\newcommand{\incpicture}[3]
{
\begin{figure}[ht]
  \vspace{10pt}
  \begin{center}
      \includegraphics[#3]{pictures/#1}
  \end{center}
%  \vspace{-30pt}
\caption{#2\label{fig:#1}}
\end{figure}
} 

\newcommand{\incborderpicture}[3]
{
\begin{figure}[ht]
  \vspace{10pt}
  \begin{center}
      \fbox{\includegraphics[#3]{pictures/#1}}
  \end{center}
  \vspace{-10pt}
\caption{#2\label{fig:#1}}
\end{figure}
} 


\newcommand{\code}[1]{{\tt #1}}

% bad
\newcommand{\vers}[1]
{
    \begin{minipage}{\textwidth}
    \begin{alltt}\normalfont
    #1
    \end{alltt}
    \end{minipage}
}



%\newcommand{\labref}[1]{{\autoref{#1}}}

% ------------------------------------------------------------------------------------------------------------------------------------------

% my index
\newcommand{\mi}[1]{\index{#1}}

% multicol index
% http://www.latex-community.org/forum/viewtopic.php?f=4&t=1735&start=0&st=0&sk=t&sd=a
\makeatletter
\renewenvironment{theindex}
  {\if@twocolumn
      \@restonecolfalse
   \else
      \@restonecoltrue
   \fi
   \setlength{\columnseprule}{0pt}
   \setlength{\columnsep}{10pt}
   \begin{multicols}{4}[\section*{\indexname}]
   \markboth{\MakeUppercase\indexname}%
            {\MakeUppercase\indexname}%
   \thispagestyle{plain}
   \setlength{\parindent}{0pt}
   \setlength{\parskip}{0pt plus 0.3pt}
   \relax
   \let\item\@idxitem}%
  {\end{multicols}\if@restonecol\onecolumn\else\clearpage\fi}
\makeatother

% cím cseréje
\addto\captionsmagyar{\renewcommand{\listfigurename}{Kották jegyzéke}}


% mwdiawiki sablonok fordításai

% szómagyarázat \szo{szó}{magyarázat}
\newcommand{\szo}[2]{\footnotesize{$\clubsuit$ #1: #2 \newline}} 

% szerzõ \szerzo{szövegíró}{zeneszerzõ}
\newcommand{\szerzo}[2]{\footnotesize{$\clubsuit$ Szöveg: #1, zene: #2 \newline}} 

% van kotta (hangjegy)
\newcommand{\vankotta}{\footnotesize{\textmusicalnote \ Ehhez a dalhoz van kotta. \newline}} 

% megjegyzes
\newcommand{\megjegyzes}[1]{\footnotesize{$\clubsuit$ Megjegyzés: #1 \newline}} 

% tanultuk (falevél) (könyv?)
\newcommand{\tanultuk}[1]{\footnotesize{$\spadesuit$ Tanultuk: #1 \newline}} 

% tánc (szív)
\newcommand{\tanc}[1]{\footnotesize{$\heartsuit$ Tánc: #1 \newline}} 

% tájegység (kockatészta)
\newcommand{\tajegyseg}[1]{\footnotesize{$\diamondsuit$ Tájegység: #1 \newline}} 

% gyűjtő
\newcommand{\gyujto}[1]{\footnotesize{$\clubsuit$ Gyűjtő: #1 \newline}} 

% helyseg \helyseg{település}{megye}
\newcommand{\helyseg}[2]{\footnotesize{$\textreferencemark$ #1 (#2 megye) \newline}} 

% \bigvarstar \Asterisk \downpitchfork \Box \triangle \surd \blacksquare \varclubsuit \varspadesuit \checkmark \diameter \maltese \smallint

% \steaming W \vardiamond U \varspade X \varclub V \varheart \SI

% ceruza: \ding{46}
\newcommand{\kotta}[2]
{
\begin{minipage}{\textwidth}
\vspace{-0.6cm}
        \phantomsection
		\addcontentsline{lof}{subsection}{#1}
%        \begin{center}
%            \includegraphics[scale=0.936]{../generated/kotta/files/#2.pdf}
            \includegraphics[scale=0.975]{generated/kotta/files/pdf/#2.pdf}
%        \end{center}
        \vspace{0.0cm}
\end{minipage}
}

\newcommand{\fejezet}[1]
{
\newpage
\addcontentsline{toc}{section}{#1}
\begin{center}
\Huge{\textbf{#1}}
\end{center}
~\newline
}

\usepackage{color}
%\usepackage{calc}

\newcounter{betu}

\setlength{\unitlength}{7.7mm}
\newcommand{\blob}{\rule[-.20\unitlength]{1.5\unitlength}{0.73\unitlength}}

\newcommand\rblob{%\thepage \firstxmark
\begin{picture}(0,0)
\put(0.5,-\value{betu}){\blob}
\put(0.77,-\value{betu}){\color{white}\textbf{\firstxmark}}
\end{picture}}

\newcommand\lblob{%\thepage \firstxmark
\begin{picture}(0,0)
\put(-2,-\value{betu}){\blob}
\put(-1.22,-\value{betu}){\color{white}\textbf{\firstxmark}}
\end{picture}}

\usepackage[pdftex,pdfpagelabels]{hyperref}
\hypersetup{
%%    bookmarks=false,         % show bookmarks bar?
    unicode=true,          % non-Latin characters in Acrobat’s bookmarks
%    pdftoolbar=true,        % show Acrobat’s toolbar?
%    pdfmenubar=true,        % show Acrobat’s menu?
%    pdffitwindow=true,      % page fit to window when opened
    pdftitle={Daloskönyv},    % title
    pdfauthor={Kovács Tamás, Széles László},%,     % author
    pdfsubject={A 105. sz. IV. Béla Cserkészcsapat daloskönyve},   % subject of the document
%    pdfnewwindow=true,      % links in new window
    pdfkeywords={dal, daloskönyv, kotta, akkord, népdal, nóta, cserkész}, % list of keywords
    colorlinks=true,       % false: boxed links; true: colored links
    linkcolor=black,          % color of internal links
    citecolor=black,        % color of links to bibliography
    filecolor=black,      % color of file links
		urlcolor=black,
		pdflang=hu-HU,
		pdfpagelayout=TwoPageRight
%		plainpages=false,
%%		hypertexnames=false,
%    urlcolor=black           % color of external links
}
% temp
%\newcommand{\url}[1]{\texttt{#1}}

\newcommand{\cim}{Daloskönyv}
\newcommand{\cscs}{105. sz. IV. Béla Cscs.}
\newcommand{\cscslong}{105. számú IV. Béla Cserkészcsapat}
\newcommand{\kiado}{Nóta Benne, 2010}


\newcommand{\resetheaders}{
\fancyhead[er,ol]{\cim}
\fancyhead[el]{\includegraphics[scale=0.05]{pictures/ivbela_black.pdf}~~\footnotesize{\rightmark}}
\fancyhead[or]{\footnotesize{\rightmark}~~\includegraphics[scale=0.05]{pictures/ivbela_black.pdf}}
\fancyfoot[C]{}
\fancyfoot[el,or]{\thepage}
\fancyfoot[er,ol]{\leftmark}
}

% lblob
\newcounter{line}
\newcommand{\secname}[1]{\addtocounter{line}{1}%
\put(-2,-\value{line}){\blob}
\put(-1.22,-\value{line}){\color{white}\textbf{#1}}
}

\newcommand{\overview}{
\begin{picture}(0,0)
\secname{A}
\secname{B}
\secname{C}
\secname{D}
\secname{E}
\secname{F}
\secname{G}
\secname{H}
\secname{I}
\secname{J}
\secname{K}
\secname{L}
\secname{M}
\secname{N}
\secname{O}
\secname{P}
\secname{R}
\secname{S}
\secname{T}
\secname{U}
\secname{V}
\secname{Z}
\end{picture}
\includegraphics[scale=0.05]{pictures/space.pdf}
}

\newcommand{\dalcim}[1]{\vspace{0.1cm}\Large{\textbf{#1}}\vspace{0.0cm}}

% TODO fordítás után:
% hosszú szövegeknél a \rightmark{cím}\rightmark{}-t beszúrni a dalok végére
% \newpage-et beszúrni azoknak a fájloknak az elejére, amik új oldalra törtek, és nincs, és emiatt a megjegyzések alulra csúsztak
% test