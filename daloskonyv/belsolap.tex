
\begin{center}

~\\[\baselineskip]

{\large \cscs}
~\\[\baselineskip]

{\LARGE \textbf{\cim}}
~\\[\baselineskip]
~\\[\baselineskip]
%{\normalsize Vicának :)}
\end{center}


\newpage
%\thispagestyle{empty}
\mbox{}
\newpage

\begin{center}

~\\[\baselineskip]

{\LARGE \cscs}
~\\[\baselineskip]

{\Huge \textbf{\cim}} 
~\\[\baselineskip]
~\\[\baselineskip]
{\large Dalaink, ahogy mi ismerjük őket}
%{\large Vicának :)}

\end{center}

~\\[\baselineskip]
~\\[\baselineskip]
~\\[\baselineskip]
~\\[\baselineskip]
~\\[\baselineskip]
~\\[\baselineskip]
~\\[\baselineskip]
~\\[\baselineskip]
~\\[\baselineskip]
~\\[\baselineskip]
~\\[\baselineskip]
~\\[\baselineskip]
{\large
Ajánljuk ezt a könyvet a dunaföldvári cserkészcsapat mindenkori tagjainak, hogy örömüket leljék az éneklésben.
~\\[\baselineskip]
Mert: 
``Aki dalol, sosem fárad el!''
}
\newpage

{\large \cim}


~\\[\baselineskip]

Készítette Kovács Tamás st. (Kovi) és Széles Laci őv. a dunaföldvári \cscslong{}ban.
~\\[\baselineskip]
Tartalmaz: \input{generated/files/tartalmaz} \newline
~\\[\baselineskip]
\url{http://cserkesz.hu/105/daloskonyv}
%~\\[\baselineskip]
%\url{http://iwiw.hu/pages/user/userdata.jsp?userID=96425}

%~\\[\baselineskip]
%A kottázás állása (fekete ahol van kotta):
%~\\[\baselineskip]
%%\newline
%\setlength{\unitlength}{0.54mm}
%\begin{picture}(0,0)
%\input{generated/files/kottaprogress}
%\end{picture}

%\setlength{\unitlength}{6mm}
%\color{black}

\vfill
Összesen \input{generated/files/countdal}dal található a könyvben, a dalok címe szerint névsorba rendezve. 
~\\[\baselineskip]
A csapatlogót (a címlapon jobb oldalt) és a hátsó borítón található képet Bakos András készítette.
A könyv \LaTeX~segítségével készült, Mediawikiből php-vel fordítva.
~\\[\baselineskip]
Az észrevételeket a \mbox{\href{mailto:kovianyo@gmail.com}{kovianyo@gmail.com}} vagy a \mbox{\href{mailto:szeleslaszlo@gmail.com}{szeleslaszlo@gmail.com}} címekre várjuk.

A kézirat lezárva: \today