\begin{center}

\mbox{}
\newpage

    \begin{table}[ht]
        \begin{tabular*}{\textwidth}{@{\extracolsep{\fill}}lcr}
        \includegraphics[scale=0.197]{pictures/liliomv_black.pdf} &  & \includegraphics[scale=0.39]{pictures/ivbela_black.pdf}\\
%        \includegraphics[width=0.2\textwidth]{pictures/liliomv_black.pdf} &  & \includegraphics[width=0.2\textwidth]{pictures/ivbela_black.pdf}\\
        \end{tabular*}
    \end{table}%

~\\[\baselineskip]
~\\[\baselineskip]

{\Huge \textbf{\cim}}
~\\[\baselineskip]
~\\[\baselineskip]
{\large \cscslong}
~\\[\baselineskip]
~\\[\baselineskip]
%Laura példánya. basszuskulcsban :)
\vfill

%{\large \textsc{\kiado}}
\end{center}

\newpage
%\mbox{}
%\newpage

\begin{center}
{\large
Szerkesztette:

Kovács Tamás

\begin{minipage}{\textwidth}
\vspace{0.6cm}
\end{minipage}

A borítót és a csapatlogót tervezte:

Bakos András

\begin{minipage}{\textwidth}
\vspace{0.6cm}
\end{minipage}

Zeneileg  szerkesztette:

Vörös Rózsa}
%\end{center}

~\\[\baselineskip]
~\\[\baselineskip]
~\\[\baselineskip]
~\\[\baselineskip]
~\\[\baselineskip]
~\\[\baselineskip]
{\large

Összesen \input{generated/files/countdal}dal található a könyvben.

A könyv \LaTeX~és Lilypond segítségével készült, Mediawikiből php-vel fordítva.

Az észrevételeket a \mbox{\href{mailto:kovianyo@gmail.com}{kovianyo@gmail.com}} címre várjuk.

Köszönet Regéné Viktor Zsuzsinak a második kiadás előtti alapos átnézésért.

A kézirat lezárva: \today

\begin{minipage}{\textwidth}
\vspace{0.6cm}
\end{minipage}

\url{http://105.cserkesz.hu/}
}
\vfill

%\begin{center}
{\large
Második kiadás

\the\year}
\end{center}


\newpage

\begin{center}

~\\[\baselineskip]
~\\[\baselineskip]%

%{\LARGE \cscs}
~\\[\baselineskip] %

{\Huge \textbf{\cim}} 
~\\[\baselineskip]
~\\[\baselineskip]
{\large A gitáros zenekar dalai}

\end{center}

~\\[\baselineskip] %
~\\[\baselineskip]
~\\[\baselineskip]
\begin{center}
\begin{minipage}{0.5\textwidth}
{
%,,
\begin{alltt}\normalfont\large{}
Megmondom a titkát, édesem a dalnak:
Önmagát hallgatja, aki dalra hallgat.
Mindenik embernek a lelkében dal van,
és a saját lelkét hallja minden dalban.
És akinek szép a lelkében az ének,
az hallja a mások énekét is szépnek.

                \textit{Babits Mihály: A második ének}
\end{alltt}
%''
~\\[\baselineskip] %
~\\[\baselineskip] %
~\\[\baselineskip]
~\\[\baselineskip]
}
\end{minipage}
\end{center}
{\large
Zenekarunk két fővel alakult meg 1990-ben. Az évek során sok fiatal csatlakozott az együtteshez, és mára közel húszan zenélünk együtt vasárnaponként a gitáros misén. Az elmúlt húsz évben rengeteg dalt énekeltünk, amelyeket ebbe a kottafüzetbe gyűjtöttünk össze. 

Szeretnénk ajánlani ezt a gyűjteményt azoknak, akik szívesen zenélnek és énekelnek velünk. Mindenkinek kívánjuk, hogy fedezze fel benne kedvenceit, hisz \cim.

}

\begin{minipage}{\textwidth}
\vspace{0.8cm}
\raggedleft{\large{\textit{Rózsa}}}
\end{minipage}
\normalsize

\newpage
\mbox{}
\newpage


