% maxSorLength: 50
% virtualLines: 35
% kottaSorCount: 4
% count(lines): 19
% columnCount: 1
\newpage
\cimcentered{Szűz Mária kis Jézuskát}{SzC5B1zMC3A1riakisJC3A9zuskC3A1t}
\kottastart
\kottapart{SzC5B1zMC3A1riakisJC3A9zuskC3A1t-1}
\kottapart{SzC5B1zMC3A1riakisJC3A9zuskC3A1t-2}
\kottapart{SzC5B1zMC3A1riakisJC3A9zuskC3A1t-3}
\kottapart{SzC5B1zMC3A1riakisJC3A9zuskC3A1t-4}
\kottaend
\begin{minipage}{\textwidth}
\begin{alltt}
2. Kemény jászol, hideg szalma a te ágyad,
   Kicsiny tested, a hidegtől mégsem bágyadt.
   Fejecskédet a szívemre ide zárom,
   Aludj, aludj szép csendesen gyöngyvirágom.
\end{alltt}
\vspace{0.0cm}
\versszakspacing
\end{minipage}
\begin{minipage}{\textwidth}
\begin{alltt}
3. Pásztoroknak Te vagy legszebb reménysége,
   Édesanyád féltve őrzött szeme fénye.
   Legkedvesebb Te vagy nékem a világon,
   Aludj, aludj szép csendesen, gyöngyvirágom.
\end{alltt}
\vspace{0.0cm}
\versszakspacing
\end{minipage}
\begin{minipage}{\textwidth}
\begin{alltt}
4. Csendes éj van, hideg szellő leng a pusztán,
   Amerre jár, rólad beszél kis Jézuskám
   Rólad zeng a kis madár a fenyőágon,
   Aludj, aludj szép csendesen, gyögyvirágom.
\end{alltt}
\vspace{0.0cm}
\versszakspacing
\end{minipage}
